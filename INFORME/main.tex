\documentclass[a4paper,11pt]{report}
\usepackage[T1]{fontenc}
\usepackage[utf8]{inputenc}
\usepackage{lmodern}
\usepackage[spanish]{babel}
\usepackage{graphicx}
\usepackage[top=3cm, bottom=2.5cm, inner=1.5cm, outer=2.5cm]{geometry} % Márgenes personalizados
\usepackage{float} % Permite posicionar mejor las figuras y tablas
\usepackage{amsmath} % Comandos para la escritura de fórmulas matemáticas de mayor complejidad
\usepackage{amsfonts} % Proporciona fuentes matemáticas
\usepackage{amssymb} % Proporciona símbolos matemáticos de la American Mathematical Society

\title{Digit Recognizer. \\ 
  Classify handwritten digits using the famous MNIST data.\\
  Organizacion de Datos 75.06}
\author{Joaquín Blanco. Padrón 94653.\\
  Ruben Alvarado. Padrón.\\
  Diego Ripetour. Padrón.\\
  Grupo en Kaggle: The Thompsons}

\begin{document}

\maketitle
\tableofcontents

\begin{abstract}
En este informe se presentara el diseño elegido para realizar el reconocimiento de dígitos manuscritos. Nuestra propuesta comprende una combinación de algoritmos de clasificación simples, los cuales fueron testeados de forma independiente de tal manera que se puedan determinar aportes, limitaciones y posibles modificaciones al conjunto de datos que mejoren los resultados o aminoren el uso de recursos.
\end{abstract}
%include == include en c
%Es preferible que cada cual tenga que trabajar en su parte
%Sin tener que lidiar con el resto. A menos que así lo quiera
\chapter{Análisis Inicial de los datos.}
Previamente al diseño de la solucion, se decidio realizar un analisis de los tados de entrenamiento. Esto tiene como objetivo revisar


\end{document}
